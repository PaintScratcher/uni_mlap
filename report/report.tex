\documentclass{article}
\begin{document}
	\title{Machine Learning and Applications (MLAP) Open Assessment}
	\author{Exam Number: Y0070813}
	\maketitle
	\section{Task 1}
	\section{Task 2}
	\section{Task 3}
	\subsection{1. Qualitative Description of the algorithms}
		\subsubsection{Locally Linear Embedding}
LLE first generates a list of neighbours for each data-point on the manifold. Then, using the assumption that the neighbourhood lies on a locally linear plane, it generates a series of weights for each of these neighbours such that the data-point can be re-constructed as a linear combination of its neighbours. This is also called creating a barycentric coordinate for the data-point based on its neighbours. These weights are then used in the lower dimensional space to re-construct the data-point. For both the weight generation and point re-construction phases, an error function is minimised to produce an optimal solution.
		\subsubsection{Laplacian eigenmaps}
In this algorithm, we first generate a neighbourhood graph. Then a weight is applied to each edge, using one of two methods, either simply applying a weight of 1 for each edge (simple-minded) or by using a heat kernel. A Laplacian matrix is then generated from the weight matrix and its diagonal. By solving the generalized eigenvector problem for this Laplacian matrix and the weight matrix you find several solutions. These solutions are then ordered my ascending sizes of the eigenvalues. After removing the first eigenvalue, which will be zero, each of these solutions can be used as an embedding dimension for each data-point.
		\subsubsection{Isomap}
		Isomap can be thought of as an extension of Multi Dimensional Scaling, as it attempts to plot every data point in a lower dimensional space such that the high dimensional distances between data-points are maintained as best as possible. The distance metric is an estimate of geodesic distance between points over the manifold, generated by summing the euclidean distance between nodes on the shortest path between the nodes over a neighbourhood graph. Step 1 generates the neighbourhood graph, either with k-nearest neighbours or fixed radius, then the shortest paths between nodes are calculated and finally the estimated geodesic distance is used to apply MDS.

First, a neighbourhood graph is created either using k nearest neighbours or a fixed radius, then the geodes

 In a basic case of standard MDS the distance metric can simply be euclidian distances between points. However, when looking at non-linear data that lies on a manifold we need a better distance metric that better descripes the manifold, as euclidian distances would leave a large amount of residual variance. Isomap uses an estimate of the geodesic distance between points over the manifold. 
 When looking at non-linear data that lies on a manifold we cannot use euclidian distances as this would leave a large amount of residual variance.
	\subsection{2. Comparing themethods for embedding data into vector space}
	\subsection{3. Describing the methods mathematically}
	\subsection{4. Description of a paper}
	
\end{document}